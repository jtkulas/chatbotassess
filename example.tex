\documentclass[
  jou,
  floatsintext,
  longtable,
  nolmodern,
  notxfonts,
  notimes,
  colorlinks=true,linkcolor=blue,citecolor=blue,urlcolor=blue]{apa7}

\usepackage{amsmath}
\usepackage{amssymb}



\usepackage[bidi=default]{babel}
\babelprovide[main,import]{english}


% get rid of language-specific shorthands (see #6817):
\let\LanguageShortHands\languageshorthands
\def\languageshorthands#1{}

\RequirePackage{longtable}
\RequirePackage{threeparttablex}

\makeatletter
\renewcommand{\paragraph}{\@startsection{paragraph}{4}{\parindent}%
	{0\baselineskip \@plus 0.2ex \@minus 0.2ex}%
	{-.5em}%
	{\normalfont\normalsize\bfseries\typesectitle}}

\renewcommand{\subparagraph}[1]{\@startsection{subparagraph}{5}{0.5em}%
	{0\baselineskip \@plus 0.2ex \@minus 0.2ex}%
	{-\z@\relax}%
	{\normalfont\normalsize\bfseries\itshape\hspace{\parindent}{#1}\textit{\addperi}}{\relax}}
\makeatother




\usepackage{longtable, booktabs, multirow, multicol, colortbl, hhline, caption, array, float, xpatch}
\usepackage{subcaption}
\renewcommand\thesubfigure{\Alph{subfigure}}
\setcounter{topnumber}{2}
\setcounter{bottomnumber}{2}
\setcounter{totalnumber}{4}
\renewcommand{\topfraction}{0.85}
\renewcommand{\bottomfraction}{0.85}
\renewcommand{\textfraction}{0.15}
\renewcommand{\floatpagefraction}{0.7}

\usepackage{tcolorbox}
\tcbuselibrary{listings,theorems, breakable, skins}
\usepackage{fontawesome5}

\definecolor{quarto-callout-color}{HTML}{909090}
\definecolor{quarto-callout-note-color}{HTML}{0758E5}
\definecolor{quarto-callout-important-color}{HTML}{CC1914}
\definecolor{quarto-callout-warning-color}{HTML}{EB9113}
\definecolor{quarto-callout-tip-color}{HTML}{00A047}
\definecolor{quarto-callout-caution-color}{HTML}{FC5300}
\definecolor{quarto-callout-color-frame}{HTML}{ACACAC}
\definecolor{quarto-callout-note-color-frame}{HTML}{4582EC}
\definecolor{quarto-callout-important-color-frame}{HTML}{D9534F}
\definecolor{quarto-callout-warning-color-frame}{HTML}{F0AD4E}
\definecolor{quarto-callout-tip-color-frame}{HTML}{02B875}
\definecolor{quarto-callout-caution-color-frame}{HTML}{FD7E14}

%\newlength\Oldarrayrulewidth
%\newlength\Oldtabcolsep


\usepackage{hyperref}




\providecommand{\tightlist}{%
  \setlength{\itemsep}{0pt}\setlength{\parskip}{0pt}}
\usepackage{longtable,booktabs,array}
\usepackage{calc} % for calculating minipage widths
% Correct order of tables after \paragraph or \subparagraph
\usepackage{etoolbox}
\makeatletter
\patchcmd\longtable{\par}{\if@noskipsec\mbox{}\fi\par}{}{}
\makeatother
% Allow footnotes in longtable head/foot
\IfFileExists{footnotehyper.sty}{\usepackage{footnotehyper}}{\usepackage{footnote}}
\makesavenoteenv{longtable}

\usepackage{graphicx}
\makeatletter
\newsavebox\pandoc@box
\newcommand*\pandocbounded[1]{% scales image to fit in text height/width
  \sbox\pandoc@box{#1}%
  \Gscale@div\@tempa{\textheight}{\dimexpr\ht\pandoc@box+\dp\pandoc@box\relax}%
  \Gscale@div\@tempb{\linewidth}{\wd\pandoc@box}%
  \ifdim\@tempb\p@<\@tempa\p@\let\@tempa\@tempb\fi% select the smaller of both
  \ifdim\@tempa\p@<\p@\scalebox{\@tempa}{\usebox\pandoc@box}%
  \else\usebox{\pandoc@box}%
  \fi%
}
% Set default figure placement to htbp
\def\fps@figure{htbp}
\makeatother







\usepackage{newtx}

\defaultfontfeatures{Scale=MatchLowercase}
\defaultfontfeatures[\rmfamily]{Ligatures=TeX,Scale=1}





\title{Using LLM--powered chatbot as an administrative modality in
Psychological assessment}


\shorttitle{chatbot assessment}


\usepackage{etoolbox}








\authorsnames{Diego Figueiras,Blanca Zutano}





\affiliation{
{Ana and Blanca's University}}




\leftheader{Figueiras and Zutano}



\abstract{We create a chatbot for purposes of Psychological assessment
and contrast responses with traditional inventory responses. Large
language model powers the chatbot -- how do we train? We aim for an
optimal--point estimate with the chatbot using the metaphor of a
computerized adaptive test. Questions are tailored to respondent
trait--level. These scores are compared against scores from traditional
measures. Criterion associations are {[}better/similar/worse{]} }

\keywords{keyword1, keyword2, keyword3}

\authornote{ 

\par{       }
\par{Correspondence concerning this article should be addressed to Diego
Figueiras, Ana and Blanca's University, 1234 Capital
St., Albany, NY, USA, Email: \href{mailto:sm@example.org}{sm@example.org}}
}

\usepackage{pbalance} 
\usepackage{float}
\makeatletter
\let\oldtpt\ThreePartTable
\let\endoldtpt\endThreePartTable
\def\ThreePartTable{\@ifnextchar[\ThreePartTable@i \ThreePartTable@ii}
\def\ThreePartTable@i[#1]{\begin{figure}[!htbp]
\onecolumn
\begin{minipage}{0.5\textwidth}
\oldtpt[#1]
}
\def\ThreePartTable@ii{\begin{figure}[!htbp]
\onecolumn
\begin{minipage}{0.5\textwidth}
\oldtpt
}
\def\endThreePartTable{
\endoldtpt
\end{minipage}
\twocolumn
\end{figure}}
\makeatother


\makeatletter
\let\endoldlt\endlongtable		
\def\endlongtable{
\hline
\endoldlt}
\makeatother

\newenvironment{twocolumntable}% environment name
{% begin code
\begin{table*}[!htbp]%
\onecolumn%
}%
{%
\twocolumn%
\end{table*}%
}% end code

\urlstyle{same}



\makeatletter
\@ifpackageloaded{caption}{}{\usepackage{caption}}
\AtBeginDocument{%
\ifdefined\contentsname
  \renewcommand*\contentsname{Table of contents}
\else
  \newcommand\contentsname{Table of contents}
\fi
\ifdefined\listfigurename
  \renewcommand*\listfigurename{List of Figures}
\else
  \newcommand\listfigurename{List of Figures}
\fi
\ifdefined\listtablename
  \renewcommand*\listtablename{List of Tables}
\else
  \newcommand\listtablename{List of Tables}
\fi
\ifdefined\figurename
  \renewcommand*\figurename{Figure}
\else
  \newcommand\figurename{Figure}
\fi
\ifdefined\tablename
  \renewcommand*\tablename{Table}
\else
  \newcommand\tablename{Table}
\fi
}
\@ifpackageloaded{float}{}{\usepackage{float}}
\floatstyle{ruled}
\@ifundefined{c@chapter}{\newfloat{codelisting}{h}{lop}}{\newfloat{codelisting}{h}{lop}[chapter]}
\floatname{codelisting}{Listing}
\newcommand*\listoflistings{\listof{codelisting}{List of Listings}}
\makeatother
\makeatletter
\makeatother
\makeatletter
\@ifpackageloaded{caption}{}{\usepackage{caption}}
\@ifpackageloaded{subcaption}{}{\usepackage{subcaption}}
\makeatother

% From https://tex.stackexchange.com/a/645996/211326
%%% apa7 doesn't want to add appendix section titles in the toc
%%% let's make it do it
\makeatletter
\xpatchcmd{\appendix}
  {\par}
  {\addcontentsline{toc}{section}{\@currentlabelname}\par}
  {}{}
\makeatother

%% Disable longtable counter
%% https://tex.stackexchange.com/a/248395/211326

\usepackage{etoolbox}

\makeatletter
\patchcmd{\LT@caption}
  {\bgroup}
  {\bgroup\global\LTpatch@captiontrue}
  {}{}
\patchcmd{\longtable}
  {\par}
  {\par\global\LTpatch@captionfalse}
  {}{}
\apptocmd{\endlongtable}
  {\ifLTpatch@caption\else\addtocounter{table}{-1}\fi}
  {}{}
\newif\ifLTpatch@caption
\makeatother

\begin{document}

\maketitle


\setcounter{secnumdepth}{-\maxdimen} % remove section numbering

\setlength\LTleft{0pt}


\subsection{Borsboom philosophy (Grok
5/14/25)}\label{borsboom-philosophy-grok-51425}

Network psychometrics, as shaped by Denny Borsboom, is grounded in a
philosophical shift from traditional psychometric models, emphasizing a
network perspective on psychological phenomena. This approach draws from
causal realism, complex systems theory, and a rejection of latent
variable models as the sole explanatory framework. Below are the key
philosophical underpinnings:

\begin{itemize}
\item
  \textbf{Causal Realism}: Borsboom advocates for understanding
  psychological constructs (e.g., depression, intelligence) as networks
  of interacting components (e.g., symptoms, behaviors) rather than as
  reflections of a single latent variable. Symptoms like sadness or
  fatigue in depression are not mere indicators of an underlying cause
  but are causally connected, influencing each other directly (e.g.,
  sadness → sleep problems → fatigue). This aligns with a realist
  ontology where the network itself is the phenomenon, not a proxy for
  something unobservable.
\item
  \textbf{Rejection of Latent Variable Dogma}: Traditional psychometrics
  often assumes latent variables (e.g., a ``depression'' factor) cause
  observed behaviors. Borsboom challenges this, arguing that latent
  variables are often statistical conveniences, not necessarily real
  entities. Network psychometrics treats psychological constructs as
  emergent properties of dynamic, reciprocal interactions among
  observable variables, sidestepping the need for latent causes unless
  empirically justified.
\item
  \textbf{Complex Systems Perspective}: Inspired by systems theory,
  Borsboom views psychological phenomena as complex, self-organizing
  systems. Networks exhibit properties like feedback loops, tipping
  points, and hysteresis (e.g., a person may remain depressed even after
  an initial trigger subsides due to self-reinforcing symptom networks).
  This perspective emphasizes dynamics over static measurement, aligning
  with a process-based ontology.
\item
  \textbf{Mutualism and Emergent Phenomena}: Borrowing from biological
  and ecological models, Borsboom's mutualism suggests that
  psychological traits emerge from mutual interactions among lower-level
  components (e.g., cognitive abilities reinforcing each other). This
  contrasts with the idea of a single, top-down cause, emphasizing a
  bottom-up, interactive process where the whole (the construct) arises
  from the parts (the network).
\item
  \textbf{Pragmatic Epistemology}: Network psychometrics prioritizes
  empirical utility and predictive power over rigid adherence to
  traditional models. Borsboom emphasizes that models should be judged
  by their ability to explain and predict real-world data, encouraging
  iterative refinement of network models based on empirical findings
  rather than dogmatic assumptions about latent structures.
\item
  \textbf{Interdisciplinary Influence}: The approach integrates insights
  from graph theory, statistical mechanics, and philosophy of science
  (e.g., rejecting reification of statistical constructs). It reflects a
  broader anti-reductionist stance, acknowledging that psychological
  phenomena are multiply determined and context-dependent, resisting
  oversimplified causal stories.
\end{itemize}

In summary, Borsboom's network psychometrics is philosophically rooted
in a causal, dynamic, and emergent view of psychological phenomena,
prioritizing observable interactions over latent abstractions. It
embraces complexity, rejects reification of statistical constructs, and
aligns with a realist yet pragmatic approach to understanding the mind.
For a deeper dive, Borsboom's seminal works, like A Network Theory of
Mental Disorders (2017), articulate these ideas with empirical
applications.






\end{document}
